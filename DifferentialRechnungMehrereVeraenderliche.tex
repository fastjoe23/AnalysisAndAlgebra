\documentclass{beamer}
\usepackage[ngerman]{babel}
\usepackage{tikz}
\usepackage{pgfplots}
\pgfplotsset{compat = newest}

\usetheme{DHBW} % Wähle das gewünschte Beamer-Theme

\title{Funktionen mehrerer Veränderlicher}
\subtitle{Mathematik 1 für Wirtschaftsinformatiker}
\author{Prof. Dr. Jonas Offtermatt}
\date{\today}

\begin{document}

\begin{frame}
  \titlepage
\end{frame}

\begin{frame}
  \frametitle{Ableitung von Funktionen mehrerer Veränderlicher}
  \framesubtitle{Grundlagen}
  Funktionen mehrerer Veränderlicher ordnen jedem Punkt im Definitionsbereich einen Funktionswert zu.
  \begin{itemize}
    \item Schreibweise: $f: \mathbb{R}^n \to \mathbb{R}$
    \item Beispiele:
      \begin{itemize}
        \item $f(x, y) = x^2 + y^2$ ist eine Funktion von zwei Veränderlichen $f: \mathbb{R}^2 \to \mathbb{R}$.
        \item $g(x, y, z) = \sin(x) + \cos(y) - z^2$ ist eine Funktion von zwei Veränderlichen $f: \mathbb{R}^3 \to \mathbb{R}$.
      \end{itemize}
    \item Die Ableitung einer Funktion mehrerer Veränderlicher gibt die Änderungsrate der Funktion in einer bestimmten Richtung an.
  \end{itemize}
\end{frame}

\begin{frame}
  \frametitle{Ableitung von Funktionen mehrerer Veränderlicher}
  \framesubtitle{Partielle Ableitungen}

  Partielle Ableitungen werden verwendet, um die Ableitung einer Funktion mehrerer Veränderlicher nach einer einzelnen Veränderlichen zu berechnen.
  \begin{itemize}
    \item Die partielle Ableitung nach einer Veränderlichen wird berechnet, indem alle anderen Veränderlichen als Konstanten behandelt werden.
    \item Schreibweise: $\frac{\partial f}{\partial x}$, $\frac{\partial f}{\partial y}$, etc.
  \end{itemize}
  Beispiele:
      \begin{itemize}
        \item Für $f(x, y) = x^2 + y^2$ ist $\frac{\partial f}{\partial x} = 2x$ und $\frac{\partial f}{\partial y} = 2y$.
        \item Für $g(x, y, z) = \sin(x) + \cos(y) - z^2$ ist $\frac{\partial g}{\partial x} = \cos(x)$, $\frac{\partial g}{\partial y} = -\sin(y)$ und $\frac{\partial g}{\partial z} = -2z$.
      \end{itemize}

\end{frame}

\begin{frame}
  \frametitle{Ableitung von Funktionen mehrerer Veränderlicher}
  \framesubtitle{Totale Ableitung und Gradient}

  \begin{itemize}
    \item Die totale Ableitung einer Funktion mehrerer Veränderlicher berücksichtigt die Änderungen aller Veränderlichen gleichzeitig.
    \item Der Gradient einer Funktion ist ein Vektor, der aus den partiellen Ableitungen besteht und die Richtung des steilsten Anstiegs angibt.
    \item Schreibweise: $\nabla f = \left(\frac{\partial f}{\partial x}, \frac{\partial f}{\partial y}, \ldots\right)$
    \item Beispiele:
      \begin{itemize}
        \item Für $f(x, y) = x^2 + y^2$ ist $\nabla f = (2x, 2y)$.
        \item Für $g(x, y, z) = \sin(x) + \cos(y) - z^2$ ist $\nabla g = (\cos(x), -\sin(y), -2z)$.
      \end{itemize}
  \end{itemize}
\end{frame}

\begin{frame}
    \frametitle{Ableitung von Funktionen mehrerer Veränderlicher}
    \framesubtitle{Die Hesse-Matrix}
  
    Die Hesse-Matrix ist eine Matrix, die Informationen über die Krümmung einer Funktion mehrerer Veränderlicher liefert. Sie wird aus den zweiten partiellen Ableitungen der Funktion gebildet.
  
    \vspace{1em}
  
    Die Hesse-Matrix $H$ einer Funktion $f(x, y)$ mit den partiellen Ableitungen $\frac{\partial^2 f}{\partial x^2}$, $\frac{\partial^2 f}{\partial y^2}$ und $\frac{\partial^2 f}{\partial x \partial y}$ sieht wie folgt aus:
  
    \[
    H =
    \begin{bmatrix}
      \frac{\partial^2 f}{\partial x^2} & \frac{\partial^2 f}{\partial x \partial y} \\
      \frac{\partial^2 f}{\partial y \partial x} & \frac{\partial^2 f}{\partial y^2}
    \end{bmatrix}
    \]
  
    \vspace{1em}
  
    Die Hesse-Matrix kann verwendet werden, um die Krümmung der Funktion an bestimmten Punkten zu analysieren und auf diese Weise Extremwerte zu identifizieren.
\end{frame}

\begin{frame}
    \frametitle{Ableitung von Funktionen mehrerer Veränderlicher}
    \framesubtitle{Die Hesse-Matrix}
    \textbf{Beispiel:} Betrachten wir die Funktion $f(x, y) = x^2 + 2xy + y^2$. Die Hesse-Matrix von $f$ lautet:
  
  \[
  H =
  \begin{bmatrix}
    2 & 2 \\
    2 & 2
  \end{bmatrix}
  \]
  \end{frame}

  \begin{frame}
    \frametitle{Ableitung von Funktionen mehrerer Veränderlicher}
    \framesubtitle{Beispiel-Schaubilder}
  
    \begin{center}
        \begin{tikzpicture}
 
            \begin{axis}[title={$f(x,y)= x^2 + 2xy + y^2$}]
             
            \addplot3 [
                domain=-20:20,
                domain y = -20:20,
                samples = 20,
                samples y = 8,
                surf,
                faceted color = teal] {x^2  + 2*x*y + y^2};
             
            \end{axis}
             
      \end{tikzpicture}
    \end{center}
  \end{frame}  
  

\begin{frame}
    \frametitle{Ableitung von Funktionen mehrerer Veränderlicher}
    \framesubtitle{Beispiel-Schaubilder}
  
    \begin{center}
        \begin{tikzpicture}
 
            \begin{axis}[title={$f(x,y)= x^2 + y^2$}]
             
            \addplot3 [
                domain=-10:10,
                domain y = -10:10,
                samples = 20,
                samples y = 8,
                surf,
                faceted color = teal] {x^2 + y^2};
             
            \end{axis}
             
      \end{tikzpicture}
    \end{center}
  \end{frame}

  \begin{frame}
    \frametitle{Ableitung von Funktionen mehrerer Veränderlicher}
    \framesubtitle{Beispiel-Schaubilder}
  
    \begin{center}
        \begin{tikzpicture}
            \begin{axis}[title={$f(x,y)= \exp(-x^2-y^2)\cdot x$}]
            \addplot3[
                surf,
            ]
            {exp(-x^2-y^2)*x};
            \end{axis}
            \end{tikzpicture}
    \end{center}
  \end{frame}

  \begin{frame}
    \frametitle{Ableitung von Funktionen mehrerer Veränderlicher}
    \framesubtitle{Berechnung von Extremwerten}
  
    Um Extremwerte einer Funktion mehrerer Veränderlicher zu berechnen, gehen wir wie folgt vor:
   
    \begin{enumerate}
      \item Berechnung der partiellen Ableitungen $\frac{\partial f}{\partial x}$ und $\frac{\partial f}{\partial y}$.
      \item Bestimmung der kritischen Punkte durch Lösen der Gleichungen $\frac{\partial f}{\partial x} = 0$ und $\frac{\partial f}{\partial y} = 0$.
      \item Überprüfung der Kandidaten für Extremwerte, indem man die Hesse-Matrix verwendet.
      \item Interpretation der Ergebnisse:
        \begin{itemize}
          \item Falls die Hesse-Matrix positiv definit ist und $\frac{\partial f}{\partial x} = \frac{\partial f}{\partial y} = 0$, handelt es sich um ein lokales Minimum.
          \item Falls die Hesse-Matrix negativ definit ist und $\frac{\partial f}{\partial x} = \frac{\partial f}{\partial y} = 0$, handelt es sich um ein lokales Maximum.
          \item Falls die Hesse-Matrix indefinit ist, handelt es sich um einen Sattelpunkt.
          \item Falls die Hesse-Matrix semi-definit ist, sind weitere Untersuchungen erforderlich.
        \end{itemize}
    \end{enumerate}
  \end{frame}
  
  \begin{frame}
    \frametitle{Ableitung von Funktionen mehrerer Veränderlicher}
    \framesubtitle{Positive und negative Definitheit der Hesse-Matrix}
  
    Die Definitheit der Hesse-Matrix gibt Auskunft über die Krümmung einer Funktion mehrerer Veränderlicher an einem kritischen Punkt. Es gibt drei mögliche Fälle:
  
    \vspace{1em}
  
    \begin{itemize}
      \item \textbf{Positiv definit:} Die Hesse-Matrix $H$ ist positiv definit, wenn für alle Vektoren $v \neq 0$ gilt: $v^T H v > 0$. Dies bedeutet, dass die Funktion an diesem Punkt ein lokales Minimum aufweist.
  
      \vspace{0.5em}
  
      \item \textbf{Negativ definit:} Die Hesse-Matrix $H$ ist negativ definit, wenn für alle Vektoren $v \neq 0$ gilt: $v^T H v < 0$. Dies bedeutet, dass die Funktion an diesem Punkt ein lokales Maximum aufweist.
  
      \vspace{0.5em}
  
      \item \textbf{Indefinit:} Die Hesse-Matrix $H$ ist indefinit, wenn es Vektoren $v_1$ und $v_2$ gibt, für die $v_1^T H v_1 > 0$ und $v_2^T H v_2 < 0$ gelten. Dies bedeutet, dass die Funktion an diesem Punkt einen Sattelpunkt aufweist.
    \end{itemize}
  \end{frame}
  
\end{document}
