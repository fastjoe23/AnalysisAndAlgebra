\documentclass[fontsize=11pt, parskip=half]{scrartcl}

\usepackage{ngerman}
\usepackage[utf8]{inputenc}
\usepackage[T1]{fontenc}
\usepackage{graphicx}
\usepackage{enumitem}
\usepackage{amsmath}
\usepackage{amssymb}
\setlength{\parindent}{0em}

% set section in CM
\setkomafont{section}{\normalfont\bfseries\Large}
\makeatletter
\renewcommand\sectionlinesformat[4]{%
  \Ifstr{#1}{section}
    {\@hangfrom{\hskip #2}{#4#3}}
    {\@hangfrom{\hskip #2#3}{#4}}% original definition for subsection, subsubsection
}
\makeatother
\renewcommand\sectionformat{\enskip\thesection\autodot}

%own commands
\newcommand{\Z}{\mathbb{Z}}
\newcommand{\Q}{\mathbb{Q}}
\newcommand{\R}{\mathbb{R}}
\newcommand{\C}{\mathbb{C}}
\newcommand{\N}{\mathbb{N}}
\newcommand{\E}{\mathbb{E}}
\renewcommand{\d}{\operatorname{d}}
\newcommand{\three}[3]{\left(\begin{array}{c}#1\\#2\\#3\end{array}\right)}
\newcommand{\four}[4]{\left(\begin{array}{c}#1\\#2\\#3\\#4\end{array}\right)}

\renewcommand{\vec}[1]{\boldsymbol #1}

\begin{document}

%% Headline
\noindent
\begin{tabular}{l}
    \textbf{Analysis und Lineare Algebra} \\    
    Prof. Dr. Jonas Offtermatt
\end{tabular}
\hfill \includegraphics[width=2cm]{DHBW.pdf}\\
\rule{\textwidth}{0.5pt}

%%

%% Title
\begin{center}
    \Large
    \textbf{Übungsblatt 5 Matrizen und lineare Gleichungssysteme}
\end{center}
%%

\section{Aufgabe} Geben Sie für die folgenden Gleichungssysteme $A \vec{x}=\vec{b}$ jeweils $rg(A)$ und $rg(A,\vec{b})$ an. Was bedeutet dies für die jeweilige Anzahl an Lösungen? \\[0.2cm]
\begin{tabular}{llll}
a) & $\begin{array}{rcrcr}
    9x &- &5y &=&19 \\
     x &- &y  &=&3\\
    \end{array}$   &
b) & $\begin{array}{rcrcr}
    13x &- & 11y &=& -11 \\
    65x &- & 55y &=& -55\\
    \end{array}$   \\[0.8cm]
c) &  $\begin{array}{rcrcr}
    7x &- & y &=& 12 \\
    14x &- & 2y &=& 20\\
    \end{array}$ & &
\end{tabular}



\section{Aufgabe} F"ur welche $a,b \in \R$ ist folgendes lineares
Gleichungssystem l"osbar? Geben Sie jeweils die L"osungsmenge an.
\[\begin{array}{rcrcrcr}
 x_1 &- &2 x_2 &+ & 3x_3 &= &5 \\
2x_1 &+ &x_2 &+ &4x_3 &= &3 \\
4x_1 &- &3x_2 &+ &ax_3 &= &13 \\
x_1 &+ &3x_2 &+ & x_3  &= &b.  \end{array}
\]




\section{Aufgabe} Berechnen Sie die Inverse der Matrix $A$ und l"osen Sie damit
das Gleichungssystem $A \vec{x}=\vec{b}$ für jeden Vektor $b_i$
einmal.
\[A= \left( \begin{array}{rrr} 1 & 3 & 1 \\
-1 & 0 & 2 \\ 1 & -1 & 1 \end{array} \right) \hspace{10mm}
\vec{b_1}=\three{1}{1}{1}, \ \vec{b_2}=\three{12}{0}{4}, \
\vec{b_3}=\three{0}{3}{4} \]




\section{Aufgabe} F"ur welche $a \in \R$ ist
\[ A= \left( \begin{array}{rrr} 1 & 0 & 1 \\
0 & 1 & 1 \\ 1 & 1 & a \end{array} \right) \] invertierbar?
Berechnen Sie gegebenenfalls die Inverse.




%

\section{Aufgabe}
\begin{enumerate}
\item [a)] Berechnen Sie im Falle der Existenz die Inverse $A^{-1}$ der Matrix
$$A=
\left(\begin{array}{cc}5& 10 \\20 & 30\\ \end{array}\right).
$$
\item [b)] Gegeben ist folgende erweiterte Matrix $(A,\vec{b})$:
$$
\left(\begin{array}{cccc|c}4& 1& 0& 0& 10\\0& 3& 0& 0& 6\\0& 0& 2& 0& 2s\\0& 0& 0& 4t& 0\\
\end{array}\right).
$$
Für welche Werte der Parameter $s,t \in \R$ ist das lineare
Gleichungssystem $A\cdot \vec{x}=\vec{b}$ lösbar? Geben Sie die
Lösungsmenge an.
\end{enumerate}







\section{Aufgabe}Gegeben sei die Matrix $
 A= \left( \begin{array}{cccc}  1 & 0 & 4 & 8 \\
                                0 & 4 & 0 & 2 \\
                                2 & 0 & 2 & 1 \\
                                0 & 1 & 0 & 0 \\ \end{array}\right)
$\\[0.3cm]
sowie die Vektoren $\vec{b_1}=\four{2}{2}{1}{1}$ und $\vec{b_2}=\four{1}{-2}{-1}{1}$.\\

\begin{enumerate}
\item[a)] Zeigen Sie, dass die Ermittlung der Inversen von $A$ zur Matrix $A^{-1}$ führt, für welche gilt:\\[0.3cm]
$
 A^{-1}= \left( \begin{array}{cccc}  -\frac{1}{3} & 1 & \frac{2}{3} & -4 \\[0.1cm]
                                     0 & 0 & 0 & 1 \\[0.1cm]
                                     \frac{1}{3} & -\frac{5}{4} & -\frac{1}{6} & 5 \\[0.1cm]
                                     0 & \frac{1}{2} & 0 & -2 \\ \end{array}\right)
$\\

\item[b)] Ermitteln Sie für die beiden Vektoren $\vec{b_1}$ und $\vec{b_2}$ jeweils die Lösung $\vec{x}$ für das Gleichungssystem $A \cdot \vec{x} = \vec{b}$.
\end{enumerate}


%

\section{Aufgabe}Ein Unternehmen stellt in den drei Abteilungen "`Reparatur"',
"`Stromerzeugung"' und "`Wasserversorgung"' innerbetriebliche
Leistungen für das gesamte Unternehmen zur Verfügung.

So hat das Unternehmen z.\,B. insgesamt 200\,m$^3$ Wasser
verbraucht, davon 4\,m$^3$ in der Abteilung Reparatur und 80\,m$^3$
in der Abteilung Strom. Für die Bereitstellung der 200\,m$^3$ Wasser
sind in der Abteilung 280\,€ an Kosten angefallen. Zusätzlich
sind in dieser Abteilung die Kosten für 6 Reparaturstunden sowie die
Kosten für 100\,kWh Strom zu berücksichtigen.

Nachfolgende Tabelle fasst die entsprechenden Informationen für alle
Abteilungen zusammen:\vspace{-0.5cm}

\begin{center}
\begin{tabular}{|l|c|c|ccc|}   \hline
                                     &      Abteilungs-         &   Gesamt-         & \multicolumn{3}{c|}{Leistungsempfänger}                    \\
 Leistungsgeber                      &      kosten  &               verbrauch       &   "`Wasser"'     &"`Reparatur"'  &"`Strom"'     \\[0.1cm] \hline
 &&&&&\\[-0.4cm]
  "`Wasser"'                         &      280\,€        &   200\,m$^3$      &       0\,m$^3$           & 4\,m$^3$         & 80\,m$^3$          \\
  "`Reparatur"'                      &      792\,€        &   20\,h           &       6\,h               & 8\,h             & 4,8\,h          \\
  "`Strom"'                          &      304\,€        &   960\,kWh        &       100\,kWh           & 32\,kWh          & 0\,kWh           \\    \hline
\end{tabular}
\end{center}

Das Unternehmen möchte die innerbetrieblichen Leistungen der drei
Abteilungen verursachungsgerecht mit Verrechnungspreisen bewerten.
Es wird folglich nach je einem Verrechnungspreis für einen
Kubikmeter Wasser, für eine Reparaturstunde und für eine
Kilowattstunde Strom gesucht.

Formulieren und lösen Sie ein lineares Gleichungssystem zur
Bestimmung von leistungsgerechten Verrechnungspreisen für die drei
Abteilungen.


\end{document}