\documentclass[fontsize=11pt, parskip=half]{scrartcl}

\usepackage{ngerman}
\usepackage[utf8]{inputenc}
\usepackage[T1]{fontenc}
\usepackage{graphicx}
\usepackage{enumitem}
\usepackage{amsmath}
\usepackage{amssymb}
\setlength{\parindent}{0em}

% set section in CM
\setkomafont{section}{\normalfont\bfseries\Large}
\makeatletter
\renewcommand\sectionlinesformat[4]{%
  \Ifstr{#1}{section}
    {\@hangfrom{\hskip #2}{#4#3}}
    {\@hangfrom{\hskip #2#3}{#4}}% original definition for subsection, subsubsection
}
\makeatother
\renewcommand\sectionformat{\enskip\thesection\autodot}

%own commands
\newcommand{\Z}{\mathbb{Z}}
\newcommand{\Q}{\mathbb{Q}}
\newcommand{\R}{\mathbb{R}}
\newcommand{\C}{\mathbb{C}}
\newcommand{\N}{\mathbb{N}}
\newcommand{\E}{\mathbb{E}}
\renewcommand{\d}{\operatorname{d}}
\newcommand{\three}[3]{\left(\begin{array}{c}#1\\#2\\#3\end{array}\right)}
\newcommand{\four}[4]{\left(\begin{array}{c}#1\\#2\\#3\\#4\end{array}\right)}

\renewcommand{\vec}[1]{\boldsymbol #1}

\begin{document}

%% Headline
\noindent
\begin{tabular}{l}
    \textbf{Analysis und Lineare Algebra} \\    
    Prof. Dr. Jonas Offtermatt
\end{tabular}
\hfill \includegraphics[width=2cm]{DHBW.pdf}\\
\rule{\textwidth}{0.5pt}

%%

%% Title
\begin{center}
    \Large
    \textbf{Übungsblatt 5 Matrizen und lineare Gleichungssysteme}
\end{center}
%%

\section{Aufgabe} Geben Sie für die folgenden Gleichungssysteme $A \vec{x}=\vec{b}$ jeweils $rg(A)$ und $rg(A,\vec{b})$ an. Was bedeutet dies für die jeweilige Anzahl an Lösungen? \\[0.2cm]
\begin{tabular}{llll}
a) & $\begin{array}{rcrcr}
    9x &- &5y &=&19 \\
     x &- &y  &=&3\\
    \end{array}$   &
b) & $\begin{array}{rcrcr}
    13x &- & 11y &=& -11 \\
    65x &- & 55y &=& -55\\
    \end{array}$   \\[0.8cm]
c) &  $\begin{array}{rcrcr}
    7x &- & y &=& 12 \\
    14x &- & 2y &=& 20\\
    \end{array}$ & &
\end{tabular}



\section{Aufgabe} F"ur welche $a,b \in \R$ ist folgendes lineares
Gleichungssystem l"osbar? Geben Sie jeweils die L"osungsmenge an.
\[\begin{array}{rcrcrcr}
 x_1 &- &2 x_2 &+ & 3x_3 &= &5 \\
2x_1 &+ &x_2 &+ &4x_3 &= &3 \\
4x_1 &- &3x_2 &+ &ax_3 &= &13 \\
x_1 &+ &3x_2 &+ & x_3  &= &b.  \end{array}
\]




\section{Aufgabe} Berechnen Sie die Inverse der Matrix $A$ und l"osen Sie damit
das Gleichungssystem $A \vec{x}=\vec{b}$ für jeden Vektor $b_i$
einmal.
\[A= \left( \begin{array}{rrr} 1 & 3 & 1 \\
-1 & 0 & 2 \\ 1 & -1 & 1 \end{array} \right) \hspace{10mm}
\vec{b_1}=\three{1}{1}{1}, \ \vec{b_2}=\three{12}{0}{4}, \
\vec{b_3}=\three{0}{3}{4} \]


\section{Aufgabe}
Studierende planen eine Party und müssen Getränke einkaufen. Sie haben die Wahl zwischen Bier, Wein und Limonade. Die Preise pro Flasche sind wie folgt:

\begin{itemize}
    \item Eine Flasche Bier kostet 2 Euro.
    \item Eine Flasche Wein kostet 9 Euro.
    \item Eine Flasche Limonade kostet 1 Euro.
\end{itemize}

Insgesamt sollen 80 Flaschen gekauft werden und dafür genau 100 Euro ausgeben werden. Außerdem soll die Anzahl der Bierflaschen doppelt so groß sein wie die Anzahl der Weinflaschen.

\begin{enumerate}[label=\alph*)]
    \item Stellen Sie ein lineares Gleichungssystem auf, das die Bedingungen des Problems beschreibt.
    \item Lösen Sie das lineare Gleichungssystem und bestimmen Sie, wie viele Flaschen Bier, Wein und Limonade die Studierenden kaufen.
    \item Die Beschränkung das doppelt so viele Bierflaschen benötigt werden war doof. Wenn Sie diese weglassen, wie viele Bierflaschen können die Studierenden maximal einkaufen?
\end{enumerate}

%

\section{Aufgabe} F"ur welche $a \in \R$ ist
\[ A= \left( \begin{array}{rrr} 1 & 0 & 1 \\
0 & 1 & 1 \\ 1 & 1 & a \end{array} \right) \] invertierbar?
Berechnen Sie gegebenenfalls die Inverse.




%

\section{Aufgabe}
\begin{enumerate}
\item [a)] Berechnen Sie im Falle der Existenz die Inverse $A^{-1}$ der Matrix
$$A=
\left(\begin{array}{cc}5& 10 \\20 & 30\\ \end{array}\right).
$$
\item [b)] Gegeben ist folgende erweiterte Matrix $(A,\vec{b})$:
$$
\left(\begin{array}{cccc|c}4& 1& 0& 0& 10\\0& 3& 0& 0& 6\\0& 0& 2& 0& 2s\\0& 0& 0& 4t& 0\\
\end{array}\right).
$$
Für welche Werte der Parameter $s,t \in \R$ ist das lineare
Gleichungssystem $A\cdot \vec{x}=\vec{b}$ lösbar? Geben Sie die
Lösungsmenge an.
\end{enumerate}







\section{Aufgabe}Gegeben sei die Matrix $
 A= \left( \begin{array}{cccc}  1 & 0 & 4 & 8 \\
                                0 & 4 & 0 & 2 \\
                                2 & 0 & 2 & 1 \\
                                0 & 1 & 0 & 0 \\ \end{array}\right)
$\\[0.3cm]
sowie die Vektoren $\vec{b_1}=\four{2}{2}{1}{1}$ und $\vec{b_2}=\four{1}{-2}{-1}{1}$.\\

\begin{enumerate}
\item[a)] Zeigen Sie, dass die Ermittlung der Inversen von $A$ zur Matrix $A^{-1}$ führt, für welche gilt:\\[0.3cm]
$
 A^{-1}= \left( \begin{array}{cccc}  -\frac{1}{3} & 1 & \frac{2}{3} & -4 \\[0.1cm]
                                     0 & 0 & 0 & 1 \\[0.1cm]
                                     \frac{1}{3} & -\frac{5}{4} & -\frac{1}{6} & 5 \\[0.1cm]
                                     0 & \frac{1}{2} & 0 & -2 \\ \end{array}\right)
$\\

\item[b)] Ermitteln Sie für die beiden Vektoren $\vec{b_1}$ und $\vec{b_2}$ jeweils die Lösung $\vec{x}$ für das Gleichungssystem $A \cdot \vec{x} = \vec{b}$.
\end{enumerate}

\end{document}