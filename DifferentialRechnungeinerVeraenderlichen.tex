\documentclass{beamer}
\usepackage[ngerman]{babel}
\usepackage{tikz}
\usepackage{pgfplots}
\usepackage{booktabs}

\usetheme{DHBW}

\title{Differentialrechnung}
\subtitle{Mathematik 1 für Wirtschaftsinformatiker}
\author{Prof. Dr. Jonas Offtermatt}
\date{\today}

\begin{document}

\section{Differentialrechnung einer Veränderlichen}
\subsection{Funktionen und ihre Eigenschaften}

\begin{frame}
  \frametitle{Funktionen und ihre Eigenschaften}
  \framesubtitle{Einführung}

  \begin{itemize}
    \item Eine \textbf{Funktion} $f$ ordnet jedem Element aus einer Definitionsmenge $D$ genau ein Element aus einer Zielmenge $Z$ zu.
    \item Schreibweise: $f: D \to Z$
    \item Beispiele:
      \begin{itemize}
        \item $f(x) = 2x + 3$ ist eine lineare Funktion.
        \item $g(x) = \sin(x)$ ist eine trigonometrische Funktion.
      \end{itemize}
    \item Eigenschaften von Funktionen:
      \begin{itemize}
        \item Definitionsbereich und Wertebereich
        \item Injektivität, Surjektivität, Bijektivität
        \item Differenzierbarkeit
      \end{itemize}
  \end{itemize}
\end{frame}


\begin{frame}
    \frametitle{Funktionen und ihre Eigenschaften}
    \framesubtitle{Definitionsbereich und Wertebereich}
  
    \begin{itemize}
      \item Der \textbf{Definitionsbereich} einer Funktion ist die Menge aller Werte, für welche die Funktion definiert ist.\\
      \begin{scriptsize}
        Beispiel: Für die Funktion $f(x) = \sqrt{x}$ ist der Definitionsbereich $D = [0, \infty)$, da die Wurzel aus negativen Zahlen nicht definiert ist.
       \end{scriptsize}
      \item Der \textbf{Wertebereich} einer Funktion ist die Menge aller Funktionswerte für die Elemente im Definitionsbereich.
      \begin{scriptsize}
        Beispiel: Für die Funktion $f(x) = x^2$ ist der Wertebereich $W = [0, \infty)$, da die Funktion immer nicht-negative Werte liefert.
    \end{scriptsize}
   \end{itemize}
   \vspace*{-.5cm}
    \begin{flushright}
        \begin{tikzpicture}[scale=0.7]
          % Definitionsbereich
          \fill[blue!20] (-2.5, 0) rectangle (2.5, 4.25);
          \draw[->] (-3, 0) -- (3, 0) node[right] {$x$};
          \node[blue] at (0, -0.4) {Definitionsbereich};
    
          % Wertebereich
          \draw[red, thick, domain=-2:2] plot (\x, {\x*\x});
          \draw[->] (-3, 0) -- (3, 0);
          \draw[->] (0, -.25) -- (0, 4.25) node[above=-0.2] {$f(x)$};
          \node[red, rotate=90] at (-2.75, 2) {Wertebereich};
        \end{tikzpicture}
      \end{flushright}
  \end{frame}

  \begin{frame}
    \frametitle{Funktionen und ihre Eigenschaften}
    \framesubtitle{Injektivität, Surjektivität und Bijektivität}
  
    \begin{itemize}
      \item Eine Funktion $f$ ist \textbf{injektiv}, wenn jedem Funktionswert höchstens ein Element aus dem Definitionsbereich zugeordnet ist.\\
      Definition: $$f(x_1) \neq f(x_2)   \text{ für alle }  x_1 \neq x_2$$         
    \end{itemize}
    \begin{center}
        \begin{tikzpicture}[scale=0.8]
          \draw[->] (-2, 0) -- (2, 0) node[right] {$x$};
          \draw[->] (0, -2) -- (0, 2) node[above] {$f(x)$};
          \draw[blue, thick, domain=-1.5:1.5] plot (\x, \x);
          \node[blue] at (-2, 1.5) {injektiv};
        \end{tikzpicture}
        \begin{tikzpicture}[scale=0.8]
          \draw[->] (-2, 0) -- (2, 0) node[right] {$x$};
          \draw[->] (0, -2) -- (0, 2) node[above] {$f(x)$};
          \draw[red, thick, domain=-1.5:1.5] plot (\x, \x*\x -1.5);
          \node[red] at (1.5, 1.5) {nicht injektiv};
        \end{tikzpicture}
      \end{center}
  \end{frame}



  \begin{frame}
    \frametitle{Funktionen und ihre Eigenschaften}
    \framesubtitle{Injektivität, Surjektivität und Bijektivität}
  
    \begin{itemize}   
  \item Eine Funktion $f$ ist \textbf{surjektiv}, wenn jeder Funktionswert mindestens einem Element aus dem Zielbereich zugeordnet ist.\\
  Definition: $$\text{Für jedes } y \text{ gibt es ein } x \text{ mit } f(x) = y$$
\end{itemize}
\begin{center}
    \begin{tikzpicture}[scale=0.8]
      \draw[->] (-2, 0) -- (2, 0) node[right] {$x$};
      \draw[->] (0, -2) -- (0, 2) node[above] {$f(x)$};
      \draw[blue, thick, domain=-1.2:1.2] plot (\x, {\x^3});
      \node[blue] at (-1.5, 1.5) {surjektiv};
    \end{tikzpicture}
    \begin{tikzpicture}[scale=0.8]
        \draw[->] (-2, 0) -- (2, 0) node[right] {$x$};
        \draw[->] (0, -2) -- (0, 2) node[above] {$f(x)$};
        \draw[red, thick, domain=-1.5:1.5] plot (\x,{1.5* sin(2 * \x r)});
        \node[red] at (1.5, -1.5) {nicht surjektiv};
      \end{tikzpicture}
  \end{center}
\end{frame}    

\begin{frame}
    \frametitle{Funktionen und ihre Eigenschaften}
    \framesubtitle{Injektivität, Surjektivität und Bijektivität}
    \begin{itemize}
        \item Eine Fuktion ist \textbf{bijektiv}, wenn sie sowohl injektiv, als auch surjektiv ist.
    \end{itemize}
\end{frame}

\subsection{Grenzwerte und Stetigkeit}
\begin{frame}
  \frametitle{Funktionen und ihre Eigenschaften}
  \framesubtitle{Grenzwerte und Stetigkeit}

  Der \textbf{Grenzwert} $L$ einer Funktion $f(x)$ für $x$ gegen einen bestimmten Wert $a$ beschreibt das Verhalten der Funktion in der Nähe von $a$.
    Schreibweise: $$\lim_{x \to a} f(x) = L$$
 
\end{frame}

\begin{frame}{Beispiele für Funktionen und ihre Grenzwerte}
  \begin{columns}[c]
    \column{0.4\textwidth}
   
    \begin{tikzpicture}[scale=0.6]
      % Funktion 1: f(x) = x^2
      \draw[->] (-2.5,0) -- (2.5,0) node[right] {$x$};
      \draw[->] (0,-0.5) -- (0,4) node[above] {$f(x)=x^2$};
      \draw[domain=-2:2,smooth,variable=\x,blue] plot ({\x},{\x*\x});
      %\draw[dashed] (-2,4) node[left] {$f(x) = x^2$};
      \draw[dotted] (1.5,2.25) -- (1.5,0) node[below] {$a$};
    \end{tikzpicture}

    \vspace{1em}

    Grenzwert: $\lim\limits_{x \to a} f(x) = a^2$
    
    \column{0.4\textwidth}
   
    \begin{tikzpicture}[scale=0.6]
      % Funktion 2: g(x) = sin(x)
      \draw[->] (-1*pi,0) -- (1*pi,0) node[right] {$x$};
      \draw[->] (0,-1.5) -- (0,1.5) node[above] {$g(x)= \sin(x)$};
      \draw[domain=-1*pi:1*pi,smooth,variable=\x,red] plot ({\x},{sin(\x r)});
      
      \draw[dotted] (pi/2,1) -- (pi/2,0) node[below] {$a$};
    \end{tikzpicture}

    \vspace{1em}

    Grenzwert: $\lim\limits_{x \to a} g(x) = \sin(a)$
  \end{columns}
\end{frame}

\begin{frame}{Beispiele für Funktionen und ihre Grenzwerte}
  \begin{columns}[c]
    \column{0.5\textwidth}
    \centering
    \begin{tikzpicture}[scale=0.6]
      % Funktion 1: h(x) = 1/x
      \draw[->] (-2.5,0) -- (2.5,0) node[right] {$x$};
      \draw[->] (0,-3) -- (0,3) node[above] {$h(x)=\frac{1}{x}$};
      \draw[domain=-2:-0.3,smooth,variable=\x,blue] plot ({\x},{1/\x});
      \draw[domain=0.3:2,smooth,variable=\x,blue] plot ({\x},{1/\x});
      %\draw[dashed] (2,-1/2) node[right] {$h(x) = \frac{1}{x}$};
      \draw[dotted] (1,0) -- (1,-3) node[below] {$a$};
    \end{tikzpicture}

    \vspace{1em}

    Grenzwert:
    
    $\lim\limits_{x \to 0} h(x) = \pm \infty$

    $\lim\limits_{x \to \pm \infty} h(x) = 0$
    
    \column{0.5\textwidth}
    \centering
    \begin{tikzpicture}[scale=0.6]
      % Funktion 2: f(x) = {x, x -1 <0, 1, x>=0}
      \draw[->] (-1.7,0) -- (2.5,0) node[right] {$x$};
      \draw[->] (0,-1.7) -- (0,2) node[above] {$f(x)\begin{cases} x - 1, & x < a \\ 1, & x \geq a \end{cases}$};
      \draw[domain=-1.3:1,smooth,variable=\x,red] plot ({\x},{\x -1});
      \draw[domain=1:2.5,smooth,variable=\x,red] plot ({\x},{1});
      %\draw[dashed] (2,1) node[right] {$f(x) = \begin{cases} x, & x < 0 \\ 1, & x \geq 0 \end{cases}$};
      \draw[dotted] (1,1) -- (1,0) node[below] {$a$};
    \end{tikzpicture}

    \vspace{1em}

    Grenzwert: 
    
    $\lim\limits_{x \to a^-} f(x) = 0$, 
    
    $\lim\limits_{x \to a^+} f(x) = 1$
  \end{columns}
\end{frame}
\begin{frame}
  \frametitle{Stetigkeit}
  \textbf{Stetigkeit} einer Funktion:
  \begin{itemize}
    \item Eine Funktion $f(x)$ heißt \textbf{stetig} an der Stelle $a$, wenn gilt $$\lim_{x \to a} f(x) = f (\lim_{x \to a^-} x) =f (\lim_{x \to a^+} x) = f(a) $$  und $f(a)$ definiert ist.
    \item Eine Funktion ist \textbf{stetig auf einem Intervall}, wenn sie an jeder Stelle in diesem Intervall stetig ist.
  \end{itemize}
\end{frame}

\begin{frame}
    \frametitle{Funktionen und ihre Eigenschaften}
    \framesubtitle{Beispiele für stetige und unstetige Funktionen}
         \begin{tikzpicture}[scale=0.7]
            \draw[->] (-2, 0) -- (2, 0) node[right] {$x$};
            \draw[->] (0, -2) -- (0, 2) node[above] {$f(x)$};
            \draw[blue, thick, domain=-1.5:1.5] plot (\x,{1.5* sin(2 * \x r)});
            \node[blue] at (-1.5, 1.5) {stetig};
        \end{tikzpicture}
         \begin{tikzpicture}[scale=0.7]
            \draw[->] (-2, 0) -- (2, 0) node[right] {$x$};
            \draw[->] (0, -2) -- (0, 2) node[above] {$f(x)$};
            \draw[blue, thick, domain=-1.5:1.5] plot (\x,{\x*\x});
        \end{tikzpicture}
        \begin{tikzpicture}[scale=0.7]
            \draw[->] (-2, 0) -- (2, 0) node[right] {$x$};
            \draw[->] (0, -2) -- (0, 2) node[above] {$f(x)$};
            \draw[blue, thick, domain=-1.2:0] plot (\x,{\x^3});
            \draw[blue, thick, domain=0:0.75] plot (\x,{\x});
            \draw[blue, thick, domain=0.75:1.5] plot (\x,0.75);
        \end{tikzpicture} 

        \begin{tikzpicture}[scale=0.7]
            \draw[->] (-2, 0) -- (2, 0) node[right] {$x$};
            \draw[->] (0, -2) -- (0, 2) node[above] {$f(x)$};
            \draw[red, thick, domain=-1.5:0] plot (\x,{\x});
            \draw[red, thick, domain=0.01:1.5] plot (\x,{\x +1});
            \node[red] at (-1.5, 1) {unstetig};            
        \end{tikzpicture}
        \begin{tikzpicture}[scale=0.7]
            \draw[->] (-2, 0) -- (2, 0) node[right] {$x$};
            \draw[->] (0, -2) -- (0, 2) node[above] {$f(x)$};
            \draw[red, thick, domain=0.6:1.5] plot (\x,{1 / \x});
            \draw[red, thick, domain=-1.5:-0.6] plot (\x,{1 / \x});
         
        \end{tikzpicture}
        \begin{tikzpicture}[scale=0.7]
            \draw[->] (-2, 0) -- (2, 0) node[right] {$x$};
            \draw[->] (0, -2) -- (0, 2) node[above] {$f(x)$};
            \draw[red,thick,-] (-1.5,-1.5) -- (-0.75,-1.5);
            \draw[red,thick,-] (-0.75,-.5) -- (0.75,-.5);
            \draw[red,thick,-] (0.75,.5) -- (1.5,.5);
         
        \end{tikzpicture}
\end{frame}  

% HIer noch L'Hopital einfügen
\subsection{Ableitung}
\begin{frame}
  \frametitle{Differentialrechnung}
  \framesubtitle{Ableitungen und Ableitungsregeln}


Die \textbf{Ableitung} einer Funktion $f(x)$ an einer Stelle $x_0$ beschreibt die Steigung der Funktion an dieser Stelle.\\

Definition: $$f'(x_0)=\frac{df}{dx}(x_0)=\lim_{x \to x_0} \frac{f(x)-f(x_0)}{x - x_0}=\lim_{h \to 0}\frac{f(x_0+h)-f(x_0)}{h}$$
\begin{center}
\begin{tikzpicture}
    \draw[->] (-2, 0) -- (2, 0) node[right] {$x$};
    \draw[->] (0, -.25) -- (0, 2) node[above] {$f(x)$};
    \draw[blue, thick, domain=-1.5:1.5] plot (\x,{\x*\x});
    \draw[red,dashed,domain=0.25:1.5] plot (\x, {2* \x -1});
    \node[red] at (1.5,1) {$f'(x)$};

\end{tikzpicture} 
\end{center}   
\end{frame}

\begin{frame}
  \frametitle{Differenzierbarkeit und stetige Differenzierbarkeit}
  Eine Funktion $f$ heißt:
  \begin{itemize}
    \item \textbf{differenzierbar} an der Stelle $x_0$, wenn $f'(x_0)$ existiert.
    \item \textbf{differenzierbar} auf dem Intervall $D$, wenn $f'(x_0)$ für alle $x_0 \in D$ existiert.
    \item \textbf{stetig differenzierbar}, falls $f$ differenzierbar und $f'$ stetig ist.
  \end{itemize}
 
  \vspace{1em}

  $f'(x_0)$ existiert, falls $f'(\lim_{x\to x_0^-})$ und $f'(\lim_{x\to x_0^+})$ existieren und es gilt:
  $$f'(\lim_{x\to x_0^-}x) = f'(\lim_{x\to x_0^+}x)$$
\end{frame}

\begin{frame}
  Es gilt: Ist eine Funktion $f$ an einer Stelle $x_0$ differenzierbar, dann ist $f$ an dieser Stelle auch stetig.

  \vspace{1em}

  Achtung: Die Umkehrung gilt nicht. Es gibt durchaus stetige Funktionen, die an bestimmten Stellen nicht differenzierbar sind.

\end{frame}

\begin{frame}
  \frametitle{Differentialrechnung}
  \framesubtitle{Ableitungsregeln}

  \begin{itemize}
    \item \textbf{Konstantenregel}: Die Ableitung einer Konstanten $c$ ist $0$.
    \item \textbf{Faktorregel}: $c\cdot f(x) \longrightarrow c \cdot f'(x)$.
    \item \textbf{Potenzregel}: $f(x) = x^n \longrightarrow f'(x) = n \cdot x^{n-1}$.
    \item \textbf{Summenregel}: $f(x) + g(x) \longrightarrow f'(x) + g'(x)$.
    \item \textbf{Produktregel}: $f(x) \cdot g(x) \longrightarrow f'(x) \cdot g(x) + f(x) \cdot g'(x)$.
    \item \textbf{Quotientenregel}:  $\frac{f(x)}{g(x)} \longrightarrow \frac{f'(x) \cdot g(x) - f(x) \cdot g'(x)}{(g(x))^2}$.
    \item \textbf{Kettenregel}:  $f(g(x)) \longrightarrow f'(g(x)) \cdot g'(x)$.
  \end{itemize}
\end{frame}

\begin{frame}
  \frametitle{Differentialrechnung}
  \framesubtitle{Beispiele von elementaren Funktionen und ihren Ableitungen}

  \begin{center}
    \begin{tabular}{c|c}
      
      \textbf{Funktion} & \textbf{Ableitung} \\
      \hline
      $f(x) = x^n$ & $f'(x) = n \cdot x^{n-1}$ \\[.7ex]
      
      $f(x) = e^x$ & $f'(x) = e^x$ \\[.7ex]
      
      $f(x) = \sin(x)$ & $f'(x) = \cos(x)$ \\[.7ex]
  
      $f(x) = \cos(x)$ & $f'(x) = -\sin(x)$ \\[.7ex]
  
      $f(x) = \ln(x)$ & $f'(x) = \frac{1}{x}$ \\[.7ex]
  
      $f(x) = \frac{1}{x}$ & $f'(x) = -\frac{1}{x^2}$ \\[.7ex]
  
      $f(x) = \sqrt[n]{x}= x^{\frac{1}{n}}, n \in \mathbb{N}$ & $f'(x)=\frac{1}{n \sqrt[n]{x^{n-1}}}$
  
    \end{tabular}
  \end{center}

  
\end{frame}

\subsection{Extremwertberechnung}
\begin{frame}
  \frametitle{Differentialrechnung}
  \framesubtitle{Die Bedeutung der ersten Ableitung für die Extremwertberechnung}

Um Extremwerte einer Funktion zu finden, suchen wir nach Stellen, an denen die Steigung $0$ ist oder nicht existiert.
 
  
    \begin{tikzpicture}
      \draw[->] (-2.5, 0) -- (2.5, 0) node[right] {$x$};
      \draw[->] (0, -2) -- (0, 2) node[above] {$f(x)$};
      \draw[blue, thick, domain=-2.25:2.25] plot (\x, {0.5*\x^3 - 2*\x});
      \draw[red, dashed] (0.11, -1.5) -- (2.2, -1.5);
      \draw[red, dashed] (-2.2, 1.5) -- (-0.1, 1.5);
      \fill[black] (-1.15, 1.53) circle (1.5pt);
      \fill[black] (1.15, -1.53) circle (1.5pt);
      \node[black, above, xshift=-0.5cm] at (-1.15, 1.53) {Lokales Maximum};
      \node[black, below, xshift=0.5cm] at (1.15, -1.53) {Lokales Minimum};
    \end{tikzpicture}
    \begin{tikzpicture}
      \draw[->] (-1.75, 0) -- (1.75, 0) node[right] {$x$};
      \draw[->] (0, -2) -- (0, 2) node[above] {$f'(x)$};
      \draw[red, thick, domain=-1.5:1.5] plot (\x, {1.5*\x*\x - 2});
    \end{tikzpicture}
  
\end{frame}

\begin{frame}
  \frametitle{Differentialrechnung}
  \framesubtitle{Die Bedeutung der zweiten und dritten Ableitung für Extremwerte}
  $$f''(x) = (f'(x))'=\frac{d f'}{dx}(x)$$
  Die \textbf{zweite Ableitung} $f''(x)$ einer Funktion gibt uns Informationen über die Krümmung der Funktion an verschiedenen Stellen. Ist die Krümmung positiv, so ist die Funktion nach oben gewölbt (lokales Minimum). Ist sie negativ, so ist die Funktion nach unten gewölbt (lokales Maximum).

  \vspace{1em}

  Die \textbf{dritte Ableitung} einer Funktion gibt uns Informationen über die Änderung der Krümmung an verschiedenen Stellen.  
\end{frame}

\begin{frame}
  \frametitle{Differentialrechnung}
  \framesubtitle{Beispiel: Maximierung der Studienleistung}
    Sie möchten ihre Studienleistung maximieren und fragen sich, wie viele Stunden pro Woche Sie 
    fürs Lernen aufwenden sollten. 

    Nehmen wir an für Ihre Studienleistung in Abhängigkeit der Stunden pro Woche gelte:
    $$L(x)=-x^3+28x^2+2940x$$
  Wie viele Stunden sollten Sie also pro Woche lernen?

  \begin{flushright}
    \begin{tikzpicture}[scale=0.5]
      \begin{axis}[
        xlabel={$x$},
        ylabel={$L(x)$},
        domain=0:73,
        samples=100,
        grid=both,
        legend entries={$L(x)=-x^3+28x^2+2940x$},
        legend pos=south east,
        axis lines=left,
        enlargelimits=0.05, % Vergrößert den Bereich um das Diagramm um 10%
      ]
        \addplot[blue,ultra thick]{-x^3 + 28*x^2 + 2940*x};
      \end{axis}
    \end{tikzpicture}
  \end{flushright}
\end{frame}



\begin{frame}
  \frametitle{Differentialrechnung}
  \framesubtitle{Anwendungen der Differentialrechnung in der Wirtschaftsinformatik}

  \begin{itemize}
    \item Kostenfunktionen und Erlösfunktionen in der Wirtschaftsinformatik
    \item Marginalanalyse: Bestimmung des Grenznutzens, der Grenzkosten und des optimalen Outputs
    \item Portfolio-Optimierung: Finden des optimalen Anlageportfolios 
    \item Anwendungen in der Datenanalyse, Modellierung von Geschäftsprozessen und Entscheidungsunterstützung
  \end{itemize}
\end{frame}


\begin{frame}{Numerische Verfahren}
    \begin{itemize}
    \item Fast alle diese Anwendungen lassen sich durch Gleichungen modellieren.
    \item Lösungen dieser Gleichungen können komplex oder sogar nicht analytisch berechenbar sein.
  \end{itemize}

  $\Longrightarrow$ Solche Gleichungen können durch \textit{numerische Verfahren} gelöst werden. (Also durch einen Algorithmus.)

  \vspace*{1em}

  Für die Nullstellenbestimmung wird oft das \textbf{Newton-Verfahren} verwendet.
\end{frame}

\begin{frame}{Das Newton-Verfahren}
  Vorgehen beim Newton-Verfahren:

  \begin{enumerate}
    \item Beginnen Sie mit einer Schätzung $x_0$ der Nullstelle.
    \item Verwenden Sie die Tangente an der Stelle $x_0$ als Annäherung an die Funktion.
    \item Finden Sie den Schnittpunkt der Tangente mit der x-Achse als nächste Schätzung $x_1$.
    \item Wiederholen Sie diesen Prozess, bis eine ausreichend genaue Näherung erreicht ist.
  \end{enumerate}

  Iterationsvorschrift:

  $$
    x_{n+1} = x_n - \frac{f(x_n)}{f'(x_n)} \quad n = 0,1,2,...
  $$
  wichtig $f'(x_n)\neq 0$.
\end{frame}

\begin{frame}{Newton-Verfahren: Grafische Darstellung}
  \begin{center}
    \begin{tikzpicture}
      % Koordinatenachsen
      \draw[->] (-2,0) -- (4,0) node[below] {$x$};
      \draw[->] (0,-1) -- (0,5) node[right] {$f(x)=(x-1)^2$};
      
      % Funktion f(x)
      \draw[scale=1,domain=-1.2:3.2,smooth,variable=\x,blue] plot ({\x},{(\x-1)^2});
      % x0
      \node[circle,draw,fill,inner sep=1pt,label=below:{$x_0$}] at (3,0) {};
      % Tangente
      \uncover<2->{
        \draw[scale=1,domain=1.95:3.1,variable=\x,red] plot ({\x},{4*\x-8});
        \draw[->,dashed] (3,0) -- (3,4) node[above] {};
        \draw[] (3,4) node[right]{$f(x_0)$};
      }
      
      %x_1
      \uncover<3->{
        \node[circle,draw,fill,inner sep=1pt,label=below:{$x_1$}] at (2,0) {};
        \draw[->,dashed] (2,0) -- (2,1) node[left] {$f(x_1)$};
      }
      
      %Tangente      
      \uncover<4->{
        \draw[scale=1,domain=1.45:2.05,variable=\x,red] plot ({\x},{2*\x-3});
      }

      %x_2
      \uncover<5->{
        \node[circle,draw,fill,inner sep=1pt,label=below:{$x_2$}] at (1.5,0) {};
        \draw[->,dashed] (1.5,0) -- (1.5,0.25) node[left] {$f(x_2)$};
        \draw[scale=1,domain=1.15:1.55,variable=\x,red] plot ({\x},{\x-1.25});
      }
    \end{tikzpicture}
  \end{center}
\end{frame}

\end{document}
