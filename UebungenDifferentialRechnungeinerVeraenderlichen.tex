\documentclass[fontsize=11pt, parskip=half]{scrartcl}

\usepackage{ngerman}
\usepackage[utf8]{inputenc}
\usepackage[T1]{fontenc}
\usepackage{graphicx}
\usepackage{enumitem}
\usepackage{amsmath}
\usepackage{amssymb}
\setlength{\parindent}{0em}

% set section in CM
\setkomafont{section}{\normalfont\bfseries\Large}
\makeatletter
\renewcommand\sectionlinesformat[4]{%
  \Ifstr{#1}{section}
    {\@hangfrom{\hskip #2}{#4#3}}
    {\@hangfrom{\hskip #2#3}{#4}}% original definition for subsection, subsubsection
}
\makeatother
\renewcommand\sectionformat{\enskip\thesection\autodot}

%own commands
\newcommand{\Z}{\mathbb{Z}}
\newcommand{\Q}{\mathbb{Q}}
\newcommand{\R}{\mathbb{R}}
\newcommand{\C}{\mathbb{C}}
\newcommand{\N}{\mathbb{N}}
\newcommand{\E}{\mathbb{E}}
\renewcommand{\d}{\operatorname{d}}

\renewcommand{\vec}[1]{\boldsymbol #1}

\begin{document}

%% Headline
\noindent
\begin{tabular}{l}
    \textbf{Analysis und Lineare Algebra} \\    
    Prof. Dr. Jonas Offtermatt
\end{tabular}
\hfill \includegraphics[width=2cm]{DHBW.pdf}\\
\rule{\textwidth}{0.5pt}

%%

%% Title
\begin{center}
    \Large
    \textbf{Übungsblatt 3 Differentialrechnung einer Veränderlichen}
\end{center}
%%
\section{Aufgabe}
Überprüfen Sie folgende Funktionen auf Injektivität, Surjektivität und Bijektivität.

\begin{tabular}{ll}
    a) $\displaystyle{ f(x)=x,\quad D=\R, W=\R}$ & b) $\displaystyle{
    f(x)=x^2,\quad D=\R, W=\R}$  \\
    c) $\displaystyle{
    f(x)=x^2,\quad D=\R, W=\R^+}$ & 
    d) $\left\{ \begin{array}{ll}
        x & ,\hspace{0.1cm} x \leq 0 \\
        x+1 & ,\hspace{0.1cm} x>0
        \end{array} \right.\quad D=\R, W=\R$ 
    \end{tabular}


\section{Aufgabe}

Berechnen Sie die erste Ableitung der folgenden Funktionen: \\
\begin{tabular}{lll}
a) $\displaystyle{ f(x)=x^3(x^3-4x)}$ & b) $\displaystyle{
f(x)=\frac{13}{x^4}}$ & c) $\displaystyle{ f(x)=\sqrt[4]{x^3}}$ \\
d) $\displaystyle{ f(x)=(x^3+x)^{25}}$ & e) $\displaystyle{
f(x)=\frac{x^3+1}{(x^2-1)^2}}$ & f) $\displaystyle{ f(x)=\sqrt{x}
\cdot \ln x}$ \\ g) $\displaystyle{ f(x)=\frac{2x+3}{e^x x^3}}$ & h)
$\displaystyle{ f(x)=(ax)^x}$ & i) $\displaystyle{ f(x)=e^{x^3}\cdot
\ln x^2}$
\end{tabular}



\section{Aufgabe}%
Untersuchen Sie die Funktion $f(x)$ auf Stetigkeit und
Differenzierbarkeit.
\[ f(x)=\left\{ \begin{array}{ll}
\displaystyle{\frac{x^2}{2|x|}} & ,\hspace{0.1cm} x \not= 0 \\
0 & ,\hspace{0.1cm} x=0
\end{array} \right. \]


%\medskip
%
%\section{Aufgabe}%
%Bestimmen Sie Definitionsbereich, Nullstellen, lokale Extrema,
%Wendepunkte und Asymptoten von $f$:
%\[f(x)=\frac{2x^2-2x-4}{x-4}\]
%
%


\section{Aufgabe}%

Ein Unternehmen produziere ein Gut gemä{\ss} folgender
Produktionsfunktion:
\[x(r)=-r^3+12r^2+60r\\
(x: \mbox{Ertrag, Output } [ME_x]; r: \mbox{Input } [ME_r]).\]

Bestimmen Sie die Nullstellen, die Extremwerte und die Wendepunkte
und interpretieren Sie diese inhaltlich.



\section{Aufgabe}Gegeben ist die Funktion $f:\ \R \rightarrow \R$ mit den
Parametern $a, c \in \R$:

$$ f(x)=\left\{\begin{array}{ll} \frac{2}{3}x \quad & x\leq 3 \\[0.3cm]
           (x-a)^{\frac{1}{2}}+ c \quad & x > 3 \end{array}\right.$$

\begin{enumerate}
\item Skizzieren Sie den Verlauf der Funktion im Bereich $0 \leq x \leq 6 $ für den Fall $a = 2$ und $c = \frac{3}{2}$.
\item Bestimmen Sie alle Werte der Parameter $a$ und $c$, so dass die
Funktion $f$ an der Stelle $x_0=3$ stetig ist.
\item Bestimmen Sie alle Werte der Parameter $a$ und $c$, so dass die
Funktion $f$ an der Stelle $x_0=3$ differenzierbar ist.
\end{enumerate}



\section{Aufgabe}Gegeben sind eine Erlösfunktion $\displaystyle E(x)=x^4+2x^2$
sowie eine Kostenfunktion $\displaystyle K(x)=e^{\sqrt{x}}$ in
Abhängigkeit von einer Menge $x$.


\begin{enumerate}
 \item Wie lautet die Gewinnfunktion $\displaystyle G(x)$?
 \item Ermitteln Sie für die Gewinnfunktion die erste Ableitung $\displaystyle G'(x)$.
 \item Ermitteln Sie für die Gewinnfunktion die zweite Ableitung $\displaystyle G''(x)$.
 \item Gesucht ist das Gewinnmaximum und es ist bekannt, dass es in der Nähe von $x=600$ liegen muss. Verwenden Sie Ihre Ergebnisse
 aus b) und c), um mit Hilfe des Newton-Verfahrens das Maximum der
 Gewinnfunktion anzunähern.\\
 \emph{Hinweis: Es sind zwei Iterationen durchzuführen.}\\
\end{enumerate}

\end{document}