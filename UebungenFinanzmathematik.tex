\documentclass[fontsize=11pt, parskip=half]{scrartcl}

\usepackage{ngerman}
\usepackage[utf8]{inputenc}
\usepackage[T1]{fontenc}
\usepackage{graphicx}
\usepackage{enumitem}
\usepackage{amsmath}
\usepackage{amssymb}
\setlength{\parindent}{0em}

% set section in CM
\setkomafont{section}{\normalfont\bfseries\Large}
\makeatletter
\renewcommand\sectionlinesformat[4]{%
  \Ifstr{#1}{section}
    {\@hangfrom{\hskip #2}{#4#3}}
    {\@hangfrom{\hskip #2#3}{#4}}% original definition for subsection, subsubsection
}
\makeatother
\renewcommand\sectionformat{\enskip\thesection\autodot}

%own commands
\newcommand{\Z}{\mathbb{Z}}
\newcommand{\Q}{\mathbb{Q}}
\newcommand{\R}{\mathbb{R}}
\newcommand{\C}{\mathbb{C}}
\newcommand{\N}{\mathbb{N}}
\newcommand{\E}{\mathbb{E}}
\renewcommand{\d}{\operatorname{d}}

\renewcommand{\vec}[1]{\boldsymbol #1}

\begin{document}

%% Headline
\noindent
\begin{tabular}{l}
    \textbf{Analysis und Lineare Algebra} \\    
    Prof. Dr. Jonas Offtermatt
\end{tabular}
\hfill \includegraphics[width=2cm]{DHBW.pdf}\\
\rule{\textwidth}{0.5pt}

%%

%% Title
\begin{center}
    \Large
    \textbf{Übungsblatt 2 Finanzmathematik}
\end{center}
%%

\section{Aufgabe}
\begin{enumerate}
    \item Ein Kapital von 2.000 € wird zu einem Jahreszins von $4
    \,\%$ angelegt. Die Zinsen werden jährlich gutgeschrieben und fortan
    ebenfalls verzinst. Auf welchen Betrag ist es nach 10 Jahren
    angewachsen?
    \item Welches Kapital muss bei einem Jahreszins von $3,5 \,\%$
    angelegt werden, wenn man nach 5 Jahren (inkl. Zinseszins) einen
    Betrag von 5.000 € zur Verf"ugung haben m"ochte? \item Nach
    welcher Zeit $t$ verdoppelt sich das Kapital von 1.000 € bei
    einem Zinssatz von $3\,\%$ pro Jahr?
    \end{enumerate}
    

    
    \section{Aufgabe} 
    Ein Kapital von 11.000 € wird jährlich mit $10\,\%$
    verzinst. Welchen Betrag $A$ darf man am Ende des ersten Jahres
    abheben, damit am Ende des zweiten Jahres wieder 11.000 € auf
    dem Konto sind?
    

    
    \section{Aufgabe}
    \begin{enumerate}
    \item Ein Kapital von 1.000 € werde 10 Jahre lang zu einem
    nominellen Zinssatz von $6 \,\%$ mit Zinseszins angelegt.
     Wie hoch ist der Kontostand nach 10 Jahren bei \vspace{-0.2cm}
    \begin{itemize}
    \item j"ahrlicher Zinszahlung? 
    \item halbj"ahriger Zinszahlung?
    \item monatlicher Zinszahlung?
    \end{itemize}\vspace{-0.2cm}
    \item Wie hoch ist in 1 jeweils der effektive Jahreszinssatz
    $p_{eff}$? 
    \end{enumerate}
    

    
    \section{Aufgabe}
     Ein Kapital $K$ werde 5 Jahre lang bei einer Verzinsung zu $1
    \,\%$ je Vierteljahr, weitere 7 Jahre bei j"ahrlicher Verzinsung zu
    $3 \,\%$ und weitere 3 Jahre bei j"ahrlicher Verzinsung zu $4 \,\%$
    verzinst (jeweils mit Zinseszins). Welcher gleichbleibende
    Jahreszinssatz $p$ ergibt das gleiche Endkapital bei stetiger
    Verzinsung?
    
  
    
    \section{Aufgabe} 
 
        Herr Müller schlie{\ss}t einen Bausparvertrag über 100.000 € ab und möchte
    zum Ende eines jeden Jahres den gleichen Betrag $E$ einzahlen. Nach
    6 Jahren möchte er $40\,\%$ der Vertragssumme angespart haben. Die
    Verzinsung betrage $3\,\%$ pro Jahr. 
    \begin{enumerate}
        \item Berechnen Sie $E$! 
    \item Berechnen Sie den Betrag $E$, falls die Einzahlungen vorschüssig,
    d.h. zu Beginn eines jeden Jahres stattfinden.
    \end{enumerate}
    
  
    
    
    \section{Aufgabe} 
    Ein Unternehmen steht vor der Entscheidung, welches von drei
    Produkten A, B oder C hergestellt werden soll. Für die Herstellung
    von Produkt A ist die Anschaffung einer Maschine $M_A$, für Produkt
    B einer Maschine $M_B$ und für Produkt C einer Maschine $M_C$
    notwendig. Eine Maschine kostet in der Anschaffung jeweils 70.000
    €. Die erforderlichen liquiden Mittel zur Anschaffung einer
    Maschine stehen dem Unternehmen zur Verfügung. Mit der Anschaffung
    der Maschine $M_A$, $M_B$ bzw. $M_C$ und dem Verkauf der Produkte
    sind am Ende eines Jahres folgende Einzahlungsüberschüsse verbunden:
    \begin{center}
    \begin{tabular}{|c|c|c|c|c|c|}  \hline
       & \multicolumn{5}{|c|}{Ende des Jahres} \\
      & 1 & 2 & 3 & 4 & 5\\\hline
      $A$ & 30.000  & 27.000  & 25.000  & -       &  -  \\
      $B$ & 27.000  & 23.000  & 19.000  & 15.000  &  -   \\
      $C$ & 17.000  & 17.000  & 17.000  & 17.000  &  17.000   \\  \hline
    \end{tabular}
    \end{center}
    
    \begin{enumerate}
    \item Für welches Produkt soll sich das Unternehmen entscheiden, wenn die
    Einzahlungsüberschüsse auf einem Konto zu einem Zinssatz von 4\,\%
    angelegt werden?
    \item Wie lautet Ihre Empfehlung, wenn für die Anschaffung einer Maschine jeweils 80.000
    € zu berücksichtigen sind?
    \end{enumerate}
    

    
    \section{Aufgabe}
     Bei einem jährlichen Zinssatz von $6\,\%$ soll eine
    Grundschuld von 50.000 € durch zehn nachsch"ussige
    Jahresannuit"aten vollst"andig getilgt werden.
    \begin{enumerate}
    \item Bestimmen Sie die Jahresannuit"at. \item Wie gro"s ist bei
    dieser Jahresannuit"at die Restschuld nach f"unf Jahren?
    \end{enumerate}


    
    \section{Aufgabe}
    Herrn Schulze stehen $K_0 = 10.000$ € Kapital zur
    Verfügung. Er möchte dieses Kapital für 10 Jahre anlegen und bekommt
    hierfür zunächst zwei Anlagemöglichkeiten \textbf{A} und \textbf{B}
    vorgelegt:
    \begin{itemize}
    \item[\textbf{A}:] Unterjährige Verzinsung mit Zinseszins: Es wird monatlich der Zinssatz $p_u=0,25$\,\% gewährt. Dies gilt für die komplette Laufzeit von 10 Jahren.
    \item[\textbf{B}:] Zuwachssparen mit jährlicher Verzinsung: In den
    Jahren 1 bis 3 wird jährlich ein Zinssatz von 2\,\% gewährt, in den
    Jahren 4 bis 6 sind es jährlich 3\,\% und in den Jahren 7 bis 10
    schlie{\ss}lich 3,75\,\% pro Jahr.
    \end{itemize}
    
    \begin{enumerate}
    \item Welche Variante führt nach Ablauf der 10 Jahre zu einem höheren Gesamtkapital $K_{10}$?
    \item Welcher gleichbleibende jährliche Zinssatz müsste gewährt werden, dass nach 10 Jahren dasselbe Kapital angespart wäre wie bei Anlage \textbf{A}?
    \item Wie müsste bei stetiger Verzinsung der Zinssatz gewahlt werden, um zu einer identischen Kapitalhöhe $K_{10}$ zu kommen wie bei Anlage \textbf{A}?
    \item Weshalb muss der Wert für den gesuchten Zinssatz in 2 über dem Wert des Zinssatzes aus 3 liegen? Begründen Sie stichhaltig.\\
    \end{enumerate}
    

    
    \section{Aufgabe}
    Herr Maier möchte seiner Tochter ein Studium an einer teuren
    Privatuniversität im Ausland ermöglichen. Hierfür muss er 100.000
   € aufbringen. Seine Bank bietet ihm einen Kredit über diesen
    Betrag mit einem Jahreszinssatz von 4\% an. Jeweils zum Jahresende
    werde die Annuität (Zinsen plus Tilgung) von 10.000€ gezahlt.
    \begin{enumerate}
    \item Welche Restschuld besteht nach 10 Jahren? %
    \item Bei welcher Annuität würde die Schuld in genau 10 Jahren
    getilgt werden?
    \item Wie lange dauert es, bis die Schuld bei einer Annuität von 10.000 € vollständig getilgt
    ist?
    \end{enumerate}
    
\end{document}