\documentclass{beamer}
\usepackage{tikz}
\usepackage[ngerman]{babel}

\usetheme{DHBW} % Wähle das gewünschte Beamer-Theme

\title{Folgen und Reihen}
\subtitle{Mathematik 1 für Wirtschaftsinformatiker}
\author{Prof. Dr. Jonas Offtermatt}
\date{\today}

\begin{document}

\begin{frame}
    \frametitle{Einführung in Folgen}
  
    \textbf{Definition einer Folge}
  
    Eine Folge ist eine geordnete Liste von Zahlen, die bestimmten Regeln folgt.
  
    \vspace{0.5em}
  
    \textbf{Mathematische Definition}
  
    Eine Abbildung $(a_n)$, die jeder natürlichen Zahl $n \in \mathbb{N}$ eine reelle Zahl $a_n$ zuordnet, heißt reelle Zahlenfolge.
  
    \vspace{0.5em}
  
    \textbf{Beispiele von Folgen}
  
    \begin{itemize}
      \item $a_n = n^2$: Die quadratische Folge $1, 4, 9, 16, \ldots$
      \item $b_n = \frac{1}{n}$: Die harmonische Folge $1, \frac{1}{2}, \frac{1}{3}, \frac{1}{4}, \ldots$
      \item $c_n = (-1)^n$: Die alternierende Folge $-1, 1, -1, 1, \ldots$
    \end{itemize}
  \end{frame}
  
  \begin{frame}
    \frametitle{Arithmetische und Geometrische Folgen}
  
    \textbf{Arithmetische Folgen}
  
    Eine arithmetische Folge ist eine Folge, in der die Differenz zwischen aufeinanderfolgenden Gliedern konstant ist.
  
    \vspace{0.5em}
  
    \textbf{Definition einer arithmetischen Folge}
  
    Eine Folge $a_1, a_2, a_3, \ldots$ heißt arithmetische Folge, wenn die Differenz $d$ zwischen zwei aufeinanderfolgenden Gliedern konstant ist. Das heißt, für alle $n \geq 1$ gilt: $a_{n+1} - a_n = d$.
  
    \vspace{0.5em}
  
    \textbf{Beispiel einer arithmetischen Folge}
  
    Die Folge $2, 5, 8, 11, 14, \ldots$ ist eine arithmetische Folge mit der Differenz $d = 3$.
  
\end{frame}
  
\begin{frame}
  \frametitle{Arithmetische und Geometrische Folgen}
  
    \textbf{Geometrische Folgen}
  
    Eine geometrische Folge ist eine Folge, in der das Verhältnis zwischen aufeinanderfolgenden Gliedern konstant ist.
  
    \vspace{0.5em}
  
    \textbf{Definition einer geometrischen Folge}
  
    Eine Folge $a_1, a_2, a_3, \ldots$ heißt geometrische Folge, wenn das Verhältnis $q$ zwischen zwei aufeinanderfolgenden Gliedern konstant ist. Das heißt, für alle $n \geq 1$ gilt: $\frac{a_{n+1}}{a_n} = q$.
  
    \vspace{0.5em}
  
    \textbf{Beispiel einer geometrischen Folge}
  
    Die Folge $3, 6, 12, 24, 48, \ldots$ ist eine geometrische Folge mit dem Verhältnis $q = 2$.
  
  \end{frame}
  
  \begin{frame}
    \frametitle{Konvergenz von Folgen}
  
    \textbf{Definition der Konvergenz}
  
    Eine Folge $a_1, a_2, a_3, \ldots$ konvergiert gegen den Grenzwert $L$, wenn für jede positive Zahl $\epsilon > 0$ ein Index $n_0$ existiert, so dass für alle $n \geq n_0$ gilt: $$|a_n - L| < \epsilon$$.
    
    Schreibweise: $\lim_{n \to \infty} a_n = L$.
  
    \vspace{0.5em}
  
    \textbf{Beispiel einer konvergenten Folge}
  
    Die Folge $1, \frac{1}{2}, \frac{1}{4}, \frac{1}{8}, \ldots$ konvergiert gegen den Grenzwert $0$.
  
    \vspace{1em}
  
    \textbf{Divergenz einer Folge}
  
    Eine Folge, die nicht gegen einen Grenzwert konvergiert, wird als divergent bezeichnet.
    
    Die Folge $1, 2, 3, 4, \ldots$ ist eine divergente Folge.
  
  \end{frame}
  
  \begin{frame}
    \frametitle{Rechenregeln für Grenzwerte bei Folgen}
  
    Für konvergente Folgen gelten folgende Rechenregeln für Grenzwerte:
  
  
    \textbf{Konstantenregel} \\
    Für eine konstante Zahl $c$ gilt: $\lim\limits_{n \to \infty} c = c$
   
    \vspace{0.3em}

    \textbf{Summenregel} \\
    Für zwei konvergente Folgen $a_n$ und $b_n$ gilt: $$\lim\limits_{n \to \infty} (a_n + b_n) = \lim\limits_{n \to \infty} a_n + \lim\limits_{n \to \infty} b_n$$
 
    \vspace{0.3em}
  
    \textbf{Produktregel} \\
    Für zwei konvergente Folgen $a_n$ und $b_n$ gilt: $$\lim\limits_{n \to \infty} (a_n \cdot b_n) = \lim\limits_{n \to \infty} a_n \cdot \lim\limits_{n \to \infty} b_n$$

\end{frame}
  
\begin{frame}
  \frametitle{Rechenregeln für Grenzwerte bei Folgen}
  
    \textbf{Quotientenregel} \\
    Für zwei konvergente Folgen $a_n$ und $b_n$ mit $b_n \neq 0$ gilt: $$\lim\limits_{n \to \infty} \left(\frac{a_n}{b_n}\right) = \frac{\lim\limits_{n \to \infty} a_n}{\lim\limits_{n \to \infty} b_n}$$

    \vspace{0.3em}

    \textbf{Potenzregel} \\
    Für eine konvergente Folge $a_n$ und eine natürliche Zahl $k$ gilt: $$\lim\limits_{n \to \infty} (a_n^k) = \left(\lim\limits_{n \to \infty} a_n\right)^k$$

    \vspace{0.3em}
    \textbf{Korridor}\\
    Gilt $a_n\leq c_n \leq b_n$ und $\lim a_n = \lim b_n$, dann folgt $\lim c_n = \lim a_n$.
  \end{frame}

  \begin{frame}
    \frametitle{Reihen}
    
    Eine Reihe ist die Summe aller Glieder einer Folge $(a_n)$:
    $$s_n = a_0 + a_1 + a_2 + \ldots + a_n = \sum_{i=0}^{n} a_i$$
  
    \vspace{0.5em}
  
    \textbf{Endliche Reihen}
  
    Eine endliche Reihe ist eine Reihe, die nur eine begrenzte Anzahl von Gliedern hat. 
  
    Die endliche Reihe $1 + 2 + 3 + 4 + 5$ hat 5 Glieder, und die Summe beträgt $15$.
  
    \vspace{0.5em}
  
    \textbf{Unendliche Reihen}
  
    Eine unendliche Reihe hat eine unendliche Anzahl von Gliedern. 
    $$\sum_{i=0}^{\infty} a_i = \lim_{n \to \infty} \sum_{i=0}^{n} a_i$$.
\end{frame}

\begin{frame}
  \frametitle{Arithmetische und Geometrische Reihen}
    \textbf{Arithmetische Reihe}
    Für die Partialsummen der arithmetische Reihe $\sum_{i=0}^{n} i $ gilt:
    $$ s_n = \sum_{i=0}^{n} i = \frac{n(n+1)}{2}$$.

    \textbf{Geometrische Reihe}
    Die Partialsummen der geometrischen Reihe $\sum_{i=0}^{n} q^i $ mit $q \in\mathbb{R}$ und $q \neq 1$ lassen sich vereinfachen zu:
    $$ s_n = \sum_{i=0}^{n} q^i = \frac{1-q^{n+1}}{1 - q }$$.

\end{frame}

\begin{frame}
  \frametitle{Konvergenz von Reihen}
  Offensichtlich gilt für die arithmetische Reihe:
  $$ \lim_{n \to \infty} s_n = \lim_{n \to \infty}  \sum_{i=0}^{n} i = \lim_{n \to \infty}  \frac{n(n+1)}{2} =\infty$$
  Es gibt aber durchaus auch unendliche Reihen die konvergieren. So gilt bspw.
  $$\lim_{n \to \infty}  s_n =\lim_{n \to \infty}  \sum_{i=0}^{n} q^i = \frac{1}{1 - q }$$
  für $q \in \mathbb{R}$ und $q < 1$.
  \end{frame}
  

\end{document}