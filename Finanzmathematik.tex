\documentclass{beamer}
\usepackage[ngerman]{babel}
\usepackage{tikz}


\usetheme{DHBW}

\title{Finanzmathematik}
\subtitle{Mathematik 1 für Wirtschaftsinformatiker}
\author{Prof. Dr. Jonas Offtermatt}
\date{\today}

\begin{document}
\section{Finanzmathematik}
\subsection{Einfache Zinsrechnung}
\begin{frame}
  \frametitle{Einfache Zinsrechnung}
  
  Der einfache Zins $I$ kann mit der Formel $I = P \cdot r \cdot t$ berechnet werden, 
  wobei $P$ den Anfangskapitalbetrag, $r$ den Zinssatz und $t$ die Laufzeit in Jahren darstellt.
  
  \vspace{0.5cm}
  
  \textbf{Beispiel}: Ein Betrag von $1000$ EUR wird zu einem Zinssatz von $5\%$ für $3$ Jahre angelegt. Wie hoch ist der einfache Zins?
  
  \vspace{0.3cm}
  
  Lösung: $I = 1000 \cdot 0.05 \cdot 3 = 150$ EUR.
\end{frame}

\subsection{Zinseszinsrechnung}
\begin{frame}
  \frametitle{Zinseszinsrechnung}
  
  Bei der \textbf{Zinseszinsrechnung} werden die Zinsen auf den Anfangskapitalbetrag sowie auf die bereits angefallenen Zinsen berechnet und addiert.
  
  \vspace{0.3cm}
  
  Die Formel für den Endbetrag $A$ unter Berücksichtigung von Zinseszinsen lautet: $$A = P \cdot (1 + r)^t$$ 
  wobei $P$ den Anfangskapitalbetrag, $r$ den Zinssatz und $t$ die Laufzeit in Jahren darstellen.
  
  \vspace{0.5cm}
  
  \textbf{Beispiel}: Ein Betrag von $1000$ EUR wird zu einem jährlichen Zinssatz von $5\%$ für $3$ Jahre angelegt. Wie hoch ist der Endbetrag mit Zinseszinsen?
  
  \vspace{0.3cm}
  
  Lösung: $A = 1000 \cdot (1 + 0.05)^3 \approx 1157.63$ EUR.
\end{frame}


\subsection{Bundesschatzbrief}
\begin{frame}
  \frametitle{Bundesschatzbrief / interner Zins}

  Angenommen, Sie haben einen Bundesschatzbrief mit einem Anfangskapital von $10,000$ EUR und folgenden jährlichen Zinssätzen:
  \begin{center}
    \begin{tabular}{|c|c|c|}
      \hline
      \textbf{Jahr} & \textbf{Zinssatz (\%)} & \textbf{Zinsbetrag (€)} \\
      \hline
      1 & 2 & 200\\
      2 & 2.5 & 250 \\
      3 & 3 & 300\\
      4 & 3.5 & 350\\
      5 & 4 & 400\\
      \hline
    \end{tabular}
  \end{center}
  Um den Gesamtzins (internen Zins) zu berechnen, müssen wir die Gleichung:
  $$ A = P (1+r)^t$$
  nach $r$ auflösen, mit $A=11.500$, $P=10.000$ und $t = 5$. Ergibt: 
  $$r\approx 2,8\%$$
\end{frame}

\subsection{Unterjährliche Zinszahlungen}
\begin{frame}
  \frametitle{Jährliche und Unterjährliche Zinszahlungen}
  Wird ein Kapital $P$ $m$-mal unterjährig verzinst, so wird der Jahreszins $r$ (auch nomineller Zins genannt) durch die Anzahl
  an Zinszahlungen $m$ geteilt. Für den unterjährigen Zins gilt also $r_u = \frac{r}{m}$.

  Der Endbetrag $A$ nach $t$ Jahren ergibt sich dann als:
  \[A_t = P(1+r_u)^{m\cdot t}\]

  Der interne Jahreszins, bzw. effektive Jahreszins ergibt sich dann als:
  $$r_{eff}=(1+r_u)^m-1$$

\end{frame}

\begin{frame}
  \frametitle{Stetiger Zins - $m\to\infty$}
  Was passiert, wenn wir $m\to \infty$ gehen lassen? Also sozusagen kontinuierlich Zinsen zahlen?
  \begin{align*}
    A_t &= \lim_{m\to\infty} P(1+r_u)^{m\cdot t}=P[\lim_{m\to\infty} (1+\frac{r}{m})^{m}] ^t\\  
        &=P\cdot e^{r\cdot t}
  \end{align*}
  
  \vspace*{2cm}

  Ohne Beweis verwenden wir $\lim_{n\to\infty}(1+\frac{1}{n})^n=e$, bzw. $\lim_{n\to\infty}(1+\frac{x}{n})^n=e^x$.
\end{frame}


\subsection{Barwertmethode}
\begin{frame}
  \frametitle{Barwertmethode}
  Der Barwert $BW$ einer Investition wird berechnet, indem die zukünftigen Zahlungen abgezinst und auf den aktuellen Zeitpunkt diskontiert werden.
  
  \vspace{0.3cm}
  
  Die Formel für den Barwert lautet: $$BW = \sum_{t=1}^{n} \frac{Z_t}{(1+r)^t}$$ 
  wobei $Z_t$ die Zahlung zum Zeitpunkt $t$, $r$ der Diskontierungssatz und $n$ die Laufzeit darstellen.
  

\end{frame}


\begin{frame}
  \frametitle{Barwertmethode}
  
  \textbf{Beispiel}: Eine Investition erfordert Zahlungen von $2,000$ EUR pro Jahr für die nächsten fünf Jahre. Der Diskontierungssatz beträgt $6\%$. Wie hoch ist der Barwert der Investition?
  
  \vspace{0.3cm}
  
  Lösung:
  \begin{align*}
    BW &= \sum_{t=1}^{n} \frac{Z_t}{(1+r)^t} = \sum_{t=1}^{5} \frac{2000}{(1+0,06)^t}\\
    &= \frac{2,000}{1.06} + \frac{2,000}{1.06^2} + \frac{2,000}{1.06^3} + \frac{2,000}{1.06^4} + \frac{2,000}{1.06^5}\\  
    &\approx 8,424.73 \text{ €.}  
  \end{align*}
\end{frame}

\begin{frame}
    \frametitle{Barwertmethode}
    Die Barwertmethode ermöglicht es verschiedene Zahlungsmodelle mit zukünftigen Zahlungen miteinander zu vergleichen.

    \textbf{Beispiel}: Ein Unternehmen schafft fur die Produktion eine neue Fertigungsanlage an. Der
    Lieferant bietet uns drei Zahlungsvarianten zur Auswahl:
    \begin{itemize}
        \item sofortige Bezahlung von 90.000 €
        \item Bezahlung von 115.000 € in 5 Jahren
        \item Bezahlung von 50.000 € in 3 Jahren und weiteren 60.000 € in 6 Jahren
    \end{itemize}
    Welche Zahlungsvariante sollte, wenn ein risikoloser Zins von $4\%$ vorliegt, unter Kostenaspekten gewählt werden? Was ändert sich bei $7\%$ Zinsen?
\end{frame}



\subsection{Kreditarten}
\begin{frame}
  \frametitle{Kreditarten}
  
  Es gibt verschiedene Arten von Krediten, die von Banken und anderen Finanzinstituten angeboten werden. Einige gängige Kreditarten sind:
  
  \begin{itemize}
    \item Annuitätendarlehen
    \item Ratenkredite
    \item Baufinanzierungen
    \item Überziehungskredite
    \item Hypothekendarlehen
  \end{itemize}
  
  Jede Kreditart hat spezifische Merkmale, Zinssätze und Tilgungsmodalitäten, die bei der Auswahl eines Kredits berücksichtigt werden sollten.
\end{frame}

\subsection{Tilgungsrechnung bei Darlehen}
\begin{frame}
  \frametitle{Tilgungsrechnung bei Annuitätendarlehen}
  
  Annuitätendarlehen sind eine häufige Form von Krediten, bei denen der Kreditbetrag $K$ über einen festgelegten Zeitraum $n$ (in Jahren) in gleichbleibenden Raten $A$ (Annuitäten) zurückgezahlt wird.

  \vspace{1cm}

  Für die Restschuld $R_n$ nach $n$ Jahren gilt dann: 
  $$R_n = K \cdot q^n - A \cdot \frac{1-q^n}{1-q}$$

  mit $q=(1+r)$.
\end{frame}

\begin{frame}
  \frametitle{Tilgungsrechnung}  
  Mithilfe der Formel für die Restschuld, können nun die verschiedenen Anwendungsfälle berechnet werden.

  \vspace{1cm}
  \textbf{Beispielfragen}:
  \begin{itemize}
    \item Wie hoch muss die Annuität $A$ sein, damit ein Kredit der Höhe $K$ bei einem Zins von $r$ nach $n$ Jahren abgezahlt ist?
    \item Wie lange muss ein Kredit der Höhe $K$ mit einem Zins von $r$ und einer Annuität von $A$ abgezahlt werden?
    \item Wie hoch kann der Kreditbetrag sein, wenn der Kredit mit Zins $r$ nach $n$ Jahren durch Annuitäten in Höhe $A$ abgezahlt sein soll?
  \end{itemize}
  Für die Beantwortung der Fragen muss die Formel für die Restschuld entsprechend umgestellt werden.
\end{frame}

\begin{frame}
  \frametitle{Tilgungsrechnung}  
  \framesubtitle{Berechnungsformeln}
  \begin{itemize}
    \item Für die Annuität $A$ gilt: $$ A = K \cdot q^n \cdot \frac{1-q}{1-q^n}$$
    \item Für den Kreditbetrag $K$ gilt: $$ K = A\cdot \frac{1-q^n}{(q^n-q^{n+1})}$$
    \item Für die Laufzeit $n$ gilt: $$ n =     \frac{\ln(A)-\ln(K(1-q)+A)}{\ln(q)}$$
  \end{itemize}
\end{frame}

\subsection{Berechnung von Renten}
\begin{frame}
  \frametitle{Berechnung von Renten}
  
  Eine Rente ist eine regelmäßige Zahlung $R$ über einen bestimmten Zeitraum $n$ aus einem Anfangskapital $K$. Dabei wird davon ausgegangen, dass das Kapital am Jahresende mit $r$ verzinst wird und anschließend nachschüssig die Rentenzahlung entnommen wird.
  
  \vspace{0.5cm}

  Für das verbleibende Kapital $K_n$ nach $n$ Zinsperioden ergibt sich:
  $$R_n = K \cdot q^n - R \cdot \frac{1-q^n}{1-q}$$
  Soll das Kapital $K$ nach $n$ Rentenzahlungen aufgebraucht sein, so gilt fur den
Rentenbetrag $R$, welcher aus einem Anfangskapital $K$ genau $n$-mal nachschussig
gezahlt werden kann:
$$R = K \cdot q^n \cdot \frac{1-q}{1-q^n}$$ 

$q=(1+r)$
  
\end{frame}

\end{document}