\documentclass[fontsize=11pt, parskip=half]{scrartcl}

\usepackage{ngerman}
\usepackage[utf8]{inputenc}
\usepackage[T1]{fontenc}
\usepackage{graphicx}
\usepackage{enumitem}
\usepackage{amsmath}
\usepackage{amssymb}
\setlength{\parindent}{0em}

% set section in CM
\setkomafont{section}{\normalfont\bfseries\Large}
\makeatletter
\renewcommand\sectionlinesformat[4]{%
  \Ifstr{#1}{section}
    {\@hangfrom{\hskip #2}{#4#3}}
    {\@hangfrom{\hskip #2#3}{#4}}% original definition for subsection, subsubsection
}
\makeatother
\renewcommand\sectionformat{\enskip\thesection\autodot}

%own commands
\newcommand{\Z}{\mathbb{Z}}
\newcommand{\Q}{\mathbb{Q}}
\newcommand{\R}{\mathbb{R}}
\newcommand{\C}{\mathbb{C}}
\newcommand{\N}{\mathbb{N}}
\newcommand{\E}{\mathbb{E}}
\renewcommand{\d}{\operatorname{d}}

\renewcommand{\vec}[1]{\boldsymbol #1}

\begin{document}

%% Headline
\noindent
\begin{tabular}{l}
    \textbf{Analysis und Lineare Algebra} \\    
    Prof. Dr. Jonas Offtermatt
\end{tabular}
\hfill \includegraphics[width=2cm]{DHBW.pdf}\\
\rule{\textwidth}{0.5pt}

%%

%% Title
\begin{center}
    \Large
    \textbf{Übungsblatt 1 Folgen und Reihen}
\end{center}
%%

\section{Aufgabe}
\begin{enumerate}[label=\alph*)]
    \item Ein Sparer legt sich einen Sparplan zurecht. Seine erste Einzahlung auf
    einem Konto beträgt 500 €. Jeden weiteren Monat zahlt er 5 € mehr
    ein als im Vormonat. Berechnen Sie den Betrag B der 120. Einzahlung
    und die Summe S aller Einzahlungen (ohne irgendwelche Zinsen) von
    der ersten bis zur 120. Einzahlung.
    \item in Versicherungsnehmer bezahlt bei Vertragsabschluss 1.000 € Prämie.
    Jedes weitere Jahr bezahlt er 8 € weniger. Berechnen Sie die Summe
    S der ersten 20 Zahlungen.
\end{enumerate}


\section{Aufgabe}
Nach seinem Eintritt in das Berufsleben beschließt Herr
E.X. Studierender von seinem Monatsgehalt jeweils einen kleinen Teil
zur Seite zu legen, um sich in Zukunft ein neues Auto leisten zu
können. Er legt im ersten Monat 200 € zur Seite
und erhöht diesen Betrag jeden weiteren Monat um 2,5\,$\%$ gegenüber dem Vormonat.\\[-0.7cm]

\begin{enumerate}
\item Nach drei Jahren möchte er einen Zwischenstand über seine
Ersparnisse erstellen. Wie hoch sind diese und wie hoch ist der
Betrag, den er zuletzt zur Seite gelegt hat?\\[-0.7cm]
\item Seinen Recherchen zufolge benötigt er für den Neuwagenkauf 20.000 €. Im wie vielten Monat
muss er das letzte Mal Geld für das Auto zur Seite legen? Wie hoch muss diese letzte Ansparung ausfallen, um auf exakt 20.000 € zu kommen?\\[-0.7cm]
\end{enumerate}

\emph{Hinweis: Verwenden Sie bei den Berechnungen die entsprechende
Formel für die Reihe.}

\section{Aufgabe}
Untersuchen Sie die angegebenen Folgen auf Konvergenz und
geben Sie, falls existent, den jeweiligen Grenzwert an.\\

\begin{tabular}{lll}
a) $\displaystyle{a_n = \frac{n^3 +1}{2n^2}}$ & b) $\displaystyle{a_n = \frac{n^2 +1}{2n^2}}$ & c) $\displaystyle{a_n = \frac{2^n +3}{4^n}}$
\\[0.5cm]
\multicolumn{2}{l}{d) $\displaystyle{a_n = \sqrt{4n^2+14n+5} -2n}$}
& e) $\displaystyle{a_n = \frac{(-1)^n}{1+n}}$

\end{tabular}

\section{Aufgabe}
Berechnen Sie den Wert der folgenden Summen:\\

\begin{tabular}{lll}
a) $\displaystyle{ \sum_{k=0}^{\infty}\frac{1}{2^{k}}}$ & b)
$\displaystyle{ \sum_{k=1}^{\infty}\frac{1}{2^{k}}}$ & c)
$\displaystyle{ \sum_{n=3}^{\infty}\frac{4^n}{6^{n+2}}}$
\end{tabular}

\section{Aufgabe}
Eine kleine Brauerei hat sich zum Ziel gesetzt,
insgesamt 5.000 Liter einer neuen Biersorte herzustellen und
abzusetzen. Hierzu werden monatlich Teilmengen gebraut und verkauft.
Aufgrund des erwarteten zunehmenden Bekanntheitsgrades soll jeden
Monat $10 \,\%$ mehr als im Vormonat hergestellt werden. Im ersten
Monat werden 100 Liter der neuen Biersorte gebraut.

\begin{enumerate}[label=\alph*)]
 \item Welche Menge wird im 10. Monat hergestellt?
 \item Wie viele Liter Bier der neuen Sorte werden im ersten Jahr gebraut?
 \item In welchem Monat erreicht die Brauerei ihr Produktionsziel?
 \end{enumerate}

\section{Aufgabe}
Zeigen Sie unter Verwendung des Zusammenhangs

$$ s_n =  \displaystyle \sum_{k=1}^n a_k = n \cdot \frac{a_1 +a_n}{2}, n \in \N,$$

dass im Falle $n \in \N_0$ für die Teilsumme $(s_n)$ einer
arithmetischen Reihe gilt:

$$ s_n =  \displaystyle \sum_{k=0}^n a_k = (n+1) \cdot \frac{a_0 +
a_n}{2}
$$

\section{Aufgabe}
Zeigen Sie unter Verwendung des Zusammenhangs

$$ s_n =  \displaystyle \sum_{k=1}^n a_k = a_1 \cdot \frac{1 -
q^n}{1-q}, n \in \N,
$$

dass im Falle $n \in \N_0$ für die Teilsumme $(s_n)$ einer
geometrischen Reihe gilt:

$$ s_n =  \displaystyle \sum_{k=0}^n a_k = a_0 \cdot \frac{1 -
q^{n+1}}{1-q}
$$

\section{Aufgabe}
Sie sind an der Planung der Bestuhlung für einen neuen
Theatersaal beteiligt. Es ist zunächst vorgesehen, dass die
vorderste Zuschauerreihe aus 15 Plätzen besteht. In jeder weiteren
Reihe erhöht sich die Anzahl um 4 Plätze je Reihe, sodass in der 2.
Reihe 19 Plätze und in der 3. Reihe 23 Plätze entstehen. Der Theatersaal soll am Ende für 2.000 Gäste Platz bieten.\\[-0.5cm]

\begin{enumerate}[label=\alph*)]
\item Welche Anzahl an Plätzen weist die 25. Reihe des Theatersaals auf? 
\item Wie viele Plätze hat der Theatersaal insgesamt, wenn Sie genau 25 Reihen im Gebäude
unterbringen?
\item Wie muss die Anzahl der Plätze in der ersten Reihe verändert werden,
wenn Sie mit 25 Reihen und einer Erhöhung um 4 Plätze
je Reihe die Vorgabe von 2.000 Plätzen erfüllen möchten?
\item Wie muss der Wert gewählt werden, um den sich die Platzanzahl je Reihe erhöht, wenn es stattdessen bei 15 Plätzen in der ersten Reihe bleibt und
nicht mehr als 25 Reihen möglich sind?
\end{enumerate}

\section{Aufgabe}
Gegeben ist die Zahlenfolge $$a_n = \frac{4}{5^{n}}, n \in
\N.$$

\begin{enumerate}
\item Welche Werte nehmen das zweite und das dritte Folgenglied an?
\item Zeigen Sie, dass es sich bei $a_n$ um eine geometrische Folge handelt.
\item Aus der Zahlenfolge $a_n$ lässt sich die zugehörige $n$-te Teilsumme $$ s_n = \sum_{k=1}^{n} a_k $$ bilden. Welchen Wert nimmt diese Summe für $n=5$ an?
\item Zeigen Sie allgemein, dass für die Reihe $s = \lim_{n \rightarrow \infty}s_n$ gilt:
$$ s = \sum_{k=1}^{\infty} \frac{4}{5^{k-1}} = \frac{4}{5^0} + \frac{4}{5^1} + \frac{4}{5^2} + \ldots = 5 $$ \\[-0.5cm]
\end{enumerate}

\end{document}